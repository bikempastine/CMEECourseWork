\documentclass[11pt]{article}
\usepackage{graphicx}
\usepackage[margin=1cm]{geometry}
\graphicspath{ {../results/} }


\title{Is Florida getting warmer?}
\date{\vspace{-2cm}}

\begin{document}
  \maketitle
  
Global climate change altered the average climate in many places during the 20th century. 
A statistical investigation reveals that the average annual temperature increased in Key West, 
Florida during the 20th century. 
This increase is highly unlikely (p-value $<$ 0.05) to have occured if 
there were no correlation between temperatures and time.

\begin{figure}[h!]
  \centering
  \includegraphics[scale=0.4]{temp.png}
  %\caption{Temperatures in Key West, Florida in the 20th century}
  \label{fig:temps}
\end{figure}

There is a positive correlation between time and temperature as can be seen in Figure 1.
This means that the average annual temperature increased in the 20th century. 
To attribute this trend to a true warming rather than chance, statistical significance of the 
correlation coefficient is tested. If the correlation is insignificant,
the correlation can be attributed to chance coincedance rather than to climate change.

To calculate the p-value, a permutation analyis is performed.
A standard t-test is not appropriate because time series data is not independent.
The temperature data is ordered randomly using the 'sample' function in R 100,000 times. 
The correlation coeffiecent of each randomized time series is computed to 
produce a probability distriution for the correlation coefficients. 
The observed correlation coefficient is compared to the probability density function.

\begin{figure}[h!]
  \centering
  \includegraphics[scale = 0.4]{density.png}
  %\caption{Distribution of generated correlation coefficients}
  \label{fig:distribution}
\end{figure}

A corrrelation coefficient as high as 0.53 did not occur at all in all 100,000 random permutations of the 
time series as can be seen in Figure 2. This means that such a high 
correlation between time and temperature is highly unlikely (p-value $<$$<$ 0.05) to have occured by chance. 
It can be concluded that Key West, Florida got warmer in the 20th century.



\end{document}